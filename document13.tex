\documentclass[journal,12pt,twocolumn]{IEEEtran}
%
\usepackage{setspace}
\usepackage{gensymb}
%\doublespacing
\singlespacing

%\usepackage{graphicx}
%\usepackage{amssymb}
%\usepackage{relsize}
\usepackage[cmex10]{amsmath}
%\usepackage{amsthm}
%\interdisplaylinepenalty=2500
%\savesymbol{iint}
%\usepackage{txfonts}
%\restoresymbol{TXF}{iint}
%\usepackage{wasysym}
\usepackage{amsthm}
%\usepackage{iithtlc}
\usepackage{mathrsfs}
\usepackage{txfonts}
\usepackage{stfloats}
\usepackage{bm}
\usepackage{cite}
\usepackage{cases}
\usepackage{subfig}
%\usepackage{xtab}
\usepackage{longtable}
\usepackage{multirow}
%\usepackage{algorithm}
%\usepackage{algpseudocode}
\usepackage{enumitem}
\usepackage{mathtools}
\usepackage{steinmetz}
\usepackage{tikz}
\usepackage{circuitikz}
\usepackage{verbatim}
\usepackage{tfrupee}
\usepackage[breaklinks=true]{hyperref}
%\usepackage{stmaryrd}
\usepackage{tkz-euclide} % loads  TikZ and tkz-base
%\usetkzobj{all}
\usetikzlibrary{calc,math}
\usepackage{listings}
    \usepackage{color}                                            %%
    \usepackage{array}                                            %%
    \usepackage{longtable}                                        %%
    \usepackage{calc}                                             %%
    \usepackage{multirow}                                         %%
    \usepackage{hhline}                                           %%
    \usepackage{ifthen}                                           %%
  %optionally (for landscape tables embedded in another document): %%
    \usepackage{lscape}     
\usepackage{multicol}
\usepackage{chngcntr}
%\usepackage{enumerate}

%\usepackage{wasysym}
%\newcounter{MYtempeqncnt}
\DeclareMathOperator*{\Res}{Res}
%\renewcommand{\baselinestretch}{2}
\renewcommand\thesection{\arabic{section}}
\renewcommand\thesubsection{\thesection.\arabic{subsection}}
\renewcommand\thesubsubsection{\thesubsection.\arabic{subsubsection}}

\renewcommand\thesectiondis{\arabic{section}}
\renewcommand\thesubsectiondis{\thesectiondis.\arabic{subsection}}
\renewcommand\thesubsubsectiondis{\thesubsectiondis.\arabic{subsubsection}}

% correct bad hyphenation here
\hyphenation{op-tical net-works semi-conduc-tor}
\def\inputGnumericTable{}                                 %%

\lstset{
%language=C,
frame=single, 
breaklines=true,
columns=fullflexible
}
%\lstset{
%language=tex,
%frame=single, 
%breaklines=true
%}

\begin{document}
%


\newtheorem{theorem}{Theorem}[section]
\newtheorem{problem}{Problem}
\newtheorem{proposition}{Proposition}[section]
\newtheorem{lemma}{Lemma}[section]
\newtheorem{corollary}[theorem]{Corollary}
\newtheorem{example}{Example}[section]
\newtheorem{definition}[problem]{Definition}
%\newtheorem{thm}{Theorem}[section] 
%\newtheorem{defn}[thm]{Definition}
%\newtheorem{algorithm}{Algorithm}[section]
%\newtheorem{cor}{Corollary}
\newcommand{\BEQA}{\begin{eqnarray}}
\newcommand{\EEQA}{\end{eqnarray}}
\newcommand{\define}{\stackrel{\triangle}{=}}

\bibliographystyle{IEEEtran}
%\bibliographystyle{ieeetr}


\providecommand{\mbf}{\mathbf}
\providecommand{\pr}[1]{\ensuremath{\Pr\left(#1\right)}}
\providecommand{\qfunc}[1]{\ensuremath{Q\left(#1\right)}}
\providecommand{\sbrak}[1]{\ensuremath{{}\left[#1\right]}}
\providecommand{\lsbrak}[1]{\ensuremath{{}\left[#1\right.}}
\providecommand{\rsbrak}[1]{\ensuremath{{}\left.#1\right]}}
\providecommand{\brak}[1]{\ensuremath{\left(#1\right)}}
\providecommand{\lbrak}[1]{\ensuremath{\left(#1\right.}}
\providecommand{\rbrak}[1]{\ensuremath{\left.#1\right)}}
\providecommand{\cbrak}[1]{\ensuremath{\left\{#1\right\}}}
\providecommand{\lcbrak}[1]{\ensuremath{\left\{#1\right.}}
\providecommand{\rcbrak}[1]{\ensuremath{\left.#1\right\}}}
\theoremstyle{remark}
\newtheorem{rem}{Remark}
\newcommand{\sgn}{\mathop{\mathrm{sgn}}}
\providecommand{\abs}[1]{\left\vert#1\right\vert}
\providecommand{\res}[1]{\Res\displaylimits_{#1}} 
\providecommand{\norm}[1]{\left\lVert#1\right\rVert}
%\providecommand{\norm}[1]{\lVert#1\rVert}
\providecommand{\mtx}[1]{\mathbf{#1}}
\providecommand{\mean}[1]{E\left[ #1 \right]}
\providecommand{\fourier}{\overset{\mathcal{F}}{ \rightleftharpoons}}
%\providecommand{\hilbert}{\overset{\mathcal{H}}{ \rightleftharpoons}}
\providecommand{\system}{\overset{\mathcal{H}}{ \longleftrightarrow}}
	%\newcommand{\solution}[2]{\textbf{Solution:}{#1}}
\newcommand{\solution}{\noindent \textbf{Solution: }}
\newcommand{\cosec}{\,\text{cosec}\,}
\providecommand{\dec}[2]{\ensuremath{\overset{#1}{\underset{#2}{\gtrless}}}}
\newcommand{\myvec}[1]{\ensuremath{\begin{pmatrix}#1\end{pmatrix}}}
\newcommand{\mydet}[1]{\ensuremath{\begin{vmatrix}#1\end{vmatrix}}}
%\numberwithin{equation}{section}
\numberwithin{equation}{subsection}
%\numberwithin{problem}{section}
%\numberwithin{definition}{section}
\makeatletter
\@addtoreset{figure}{problem}
\makeatother

\let\StandardTheFigure\thefigure
\let\vec\mathbf
%\renewcommand{\thefigure}{\theproblem.\arabic{figure}}
\renewcommand{\thefigure}{\theproblem}
%\setlist[enumerate,1]{before=\renewcommand\theequation{\theenumi.\arabic{equation}}
%\counterwithin{equation}{enumi}


%\renewcommand{\theequation}{\arabic{subsection}.\arabic{equation}}

\def\putbox#1#2#3{\makebox[0in][l]{\makebox[#1][l]{}\raisebox{\baselineskip}[0in][0in]{\raisebox{#2}[0in][0in]{#3}}}}
     \def\rightbox#1{\makebox[0in][r]{#1}}
     \def\centbox#1{\makebox[0in]{#1}}
     \def\topbox#1{\raisebox{-\baselineskip}[0in][0in]{#1}}
     \def\midbox#1{\raisebox{-0.5\baselineskip}[0in][0in]{#1}}

\vspace{3cm}


\title{Assignment 9}
\author{Jayati Dutta}





% make the title area
\maketitle

\newpage

%\tableofcontents

\bigskip

\renewcommand{\thefigure}{\theenumi}
\renewcommand{\thetable}{\theenumi}
%\renewcommand{\theequation}{\theenumi}


\begin{abstract}
This is a simple document explaining how to express a matrix by the linear combination of the rows of another matrix.
\end{abstract}

%Download all python codes 
%
%\begin{lstlisting}
%svn co https://github.com/JayatiD93/trunk/My_solution_design/codes
%\end{lstlisting}

Download all and latex-tikz codes from 
%
\begin{lstlisting}
svn co https://github.com/gadepall/school/trunk/ncert/geometry/figs
\end{lstlisting}
%


\section{Problem}
Let $A$ = $\myvec{1 & -1\\2 & 2\\1 & 0}$ and $B$ = $\myvec{3 & 1\\-4 & 4}$
Is there any matrix $C$ such that $CA$ = $B$?

\section{Explanation}
The matrix $B$ is obtained by multiplying the matrix $A$ with matrix $C$. $B$ is a $2 \times 2$ matrix and $A$ is a $3 \times 2$ matrix. so matrix $C$ must be a $2 \times 3$ matrix.
Let the matrix $C$ is:
\begin{align}
C = \myvec{a_1 & b_1 & c_1\\a_2 & b_2 & c_2}
\end{align}
So, after multiplying with $A$ matrix we get,
\begin{multline}
\myvec{a_1 & b_1 & c_1\\a_2 & b_2 & c_2}\myvec{1 & -1\\2 & 2\\1 & 0} =\\ \myvec{a_1+2b_1+c_1 & -a_1+2b_1\\a_2+2b_2+c_2 & -a_2+2b_2}  
\end{multline}

Matrix $A$ is a rectangular matrix, so pseodo inverse of matrix $A$ exists.
Now, Considering $CA$ =$B$ and by transposing both side,
\begin{align}
(CA)^T = B^T\\
\implies A^T C^T = B^T\\
\implies \myvec{1 & 2 & 1\\-1 & 2 & 0} \myvec{| & |\\c_1 & c_2\\| & |} = \myvec{3 & -4\\1 & 4}
\end{align}
% where $P$ = $A^T$.
%But $\det(P^T P) = 0$


%Now equating this result with matrix $B$:
%\begin{align}
%\myvec{a_1+2b_1+c_1 & -a_1+2b_1\\a_2+2b_2+c_2 & -a_2+2b_2}=\myvec{3 & 1\\-4 & 4}\\
%\implies  a_1+2b_1+c_1= 3\\
%\implies  a_1+2b_1 = 3 - c_1\\
%-a_1+2b_1 = 1\\
%\implies \myvec{1 & 2\\-1 & 2}\myvec{a_1\\b_1} = \myvec{3 - c_1\\1} 
%\end{align}
%Now, considering these two equations the augmented matrix is formed as:
%\begin{align}
%\myvec{1 & 2 & (3 - c_1)\\-1 & 2 & 1}\xleftrightarrow[]{R_2\leftarrow R_1+R_2}\myvec{1 & 2 & (3 - c_1)\\0 & 4 & (4-c_1)}\\
%\xleftrightarrow[]{R_2\leftarrow R_2/2}\myvec{1 & 2 & (3 - c_1)\\0 & 2 & \frac{4-c_1}{2}}\xleftrightarrow[R_2\leftarrow R_2/2]{R_1\leftarrow R_1-R_2}\\
%\myvec{1 & 0 & \frac{2-c_1}{2}\\0 & 1 & \frac{4-c_1}{4}}\\
%\implies a_1 = \frac{2-c_1}{2}\\
% b_1 = \frac{4-c_1}{4}
%\end{align}
%Now depending on the value of $c_1$, the values of $a_1$ and $b_1$ will be calculated.
%Let $c_1$ = 4, then $a_1$ = -1 and $b_1$=0.
%Similarly,
%\begin{align}
%a_2+2b_2+c_2 = -4\\
%\implies a_2+2b_2 = -4 -c_2\\
%-a_2+2b_2 = 4
%\implies \myvec{1 & 2\\-1 & 2}\myvec{a_2\\b_2} = \myvec{-4 -c_2\\4}
%\end{align}
%Now, considering these two equations the augmented matrix is formed as:
%\begin{align}
%\myvec{1 & 2 & (-4 -c_2)\\-1 & 2 & 1}\xleftrightarrow[]{R_2\leftarrow R_1+R_2}\myvec{1 & 2 & (-4 -c_2)\\0 & 4 & -c_2}\\
%\xleftrightarrow[]{R_2\leftarrow R_2/2}\myvec{1 & 2 & (-4 -c_2)\\0 & 2 & -\frac{-c_2}{2}}\xleftrightarrow[R_2\leftarrow R_2/2]{R_1\leftarrow R_1-R_2}\\
%\myvec{1 & 0 & -4-\frac{c_2}{2}\\0 & 1 & -\frac{c_2}{4}}\\
%\implies a_2 = -4-\frac{c_2}{2}\\
% b_2 = -\frac{c_2}{4}
%\end{align}
%Now depending on the value of $c_2$, the values of $a_2$ and $b_2$ will be calculated.
%Let $c_2$ = 4, then $a_2$ = -6 and $b_2$=-1.
%
%
%\section{Solution}
%So, it can be oserved that matrix $C$ exists and depending on the $c_1$ and $c_2$ values different $C$ matrix can be generated. One of the $C$ matrix is = $\myvec{-1 & 0 & 4\\-6 & -1 & 4}$ such that $CA$=$B$.
%
%\renewcommand{\theequation}{\theenumi}
%\begin{enumerate}[label=\thesection.\arabic*.,ref=\thesection.\theenumi]
%\numberwithin{equation}{enumi}
%\item Verification of the above problem using python code.\\
%\solution The  following Python code verifies the above solution.
%\begin{lstlisting}
%codes/multiplication_test.py
%\end{lstlisting}
%%%
%\end{enumerate}

\end{document}



